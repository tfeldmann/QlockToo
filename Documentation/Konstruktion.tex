% !TEX root = QlockToo.tex

\section{Konstruktion und Produktion}
\label{sec:KonstruktionFertigung}

\begin{multicols}{2}
Um die Herstellungskosten möglichst gering zu halten wird die ClockToo als eine Kombination aus Kaufteilen und eigener Produktion ausgeführt. Ziel ist eine möglichst getreue Nachbildung der Original CLOCKTWO® der Biegert~\&~Funk Manufacture GmbH \& Co. KG. 

\textbf{Das Uhrgehäuse}

Die Basis für das Uhrengehäuse bildet der IKEA-Bilderrahmen RIBBA in schwarz. Die Bilddarstellungsmaße von 50 x 50 cm eigenen sich gut, um die Buchstabenmatrix zur Geltung zu bringen. 

{
\centering \includegraphics[width=0.75\columnwidth]{Abbildungen/Konstruktion/Ribba02} %Bilderrahmen

}
In dem besonders tiefen Rahmen,  Tiefe~=~4,5~cm, kann die komplette LED-Matrix inklusive der Elektronik untergebracht werden. Es sind lediglich keine Modifikationen notwendig.

{
\centering \includegraphics[width=0.5\columnwidth]{Abbildungen/Konstruktion/Taster03} 

}
Zum Hinausführen des Stromkabels und zur Anbringung der Taster muss der Rahmen an den entsprechenden Stellen aufgebohrt werden. 

{
\centering \includegraphics[width=0.5\columnwidth]{Abbildungen/Konstruktion/Stecker01}

}



\textbf{Buchstabenfolie}

Zur Darstellung der Zeit in Worten wird eine schwarze Folie mit einer negativen Buchstabenmatrix auf eine Plexiglasscheibe geklebt. Die Anordnung der Buchstaben entspricht der Original CLOCKTWO®. 

{
\centering\includegraphics[width=0.75\columnwidth]{Abbildungen/Konstruktion/Buchstaben}

}
Mit der gewählten Buchstabenmatrix ist die Uhr auf eine deutsche Anzeige festgelegt. Die Original CLOCKTWO® CLASSIC beherrscht zwölf Sprachen. Durch Austauschen des Frontcovers kann die gewünschte Sprache eingestellt werden. 

\textbf{LED-Matrix}

Die LED-Matrix wird aus einer quadratischen MDF-Platte gefertigt. Abbildung ~\ref{fig:Spanplatte} zeigt die Fertigungs- bzw. Bearbeitungszeichnung der Spanplatte. Diese wird entsprechend der vorgegebenen Buchstabenmatrix -  10~x~11 Buchstaben - gebohrt und gesenkt, sodass die LEDs die kompletten Buchstaben ausleuchten können. 

{
\centering\includegraphics[width=0.8\columnwidth]{Abbildungen/Konstruktion/Platte02}

}
Für den Lichtsensor und für die Minuten-LEDs werden weitere Löcher am oberen Rand und in den vier Ecken gebohrt. Zusätzlich wird ein Ausschnitt zur Aufnahme der Platine aus der Platte gefräst. 

{
\centering\includegraphics[width=0.85\columnwidth]{Abbildungen/Konstruktion/Platte03}

}
Die LEDs werden entsprechend hinter den Löchern positioniert, verdrahtet und mit Heißkleber fixiert. Bei der Erstellung des LED-Gitters ist darauf zu achten, dass die LED-Beinchen, Anode und Kathode, immer in die selbe Richtung zeigen. 

{
\centering\includegraphics[width=0.85\columnwidth]{Abbildungen/Konstruktion/LED01}

}
Die Verkabelung erfolgt möglichst direkt, einfach und übersichtlich mit Flachbandkabeln. 

{
\centering\includegraphics[width=0.85\columnwidth]{Abbildungen/Konstruktion/Platte04}

}

\textbf{LEDs}

Ohne eine zusätzliche Streuung kann man die LEDs hinter den jeweiligen Buchstaben deutlich erkennen. Eine vollständige Ausleuchtung ist nicht möglich. 

{
\centering\includegraphics[width=0.85\columnwidth]{Abbildungen/Konstruktion/LED03}

}
Um eine möglichst breite Streuung des LED-Lichtes zu erzielen, wird hinter die Buchstabenfolie eine zweite Diffusorfolie  geklebt. Trotz der doppelten Folie ist die Streuung des Lichtes jedoch nicht ausreichend. 

{
\centering\includegraphics[width=0.5\columnwidth]{Abbildungen/Konstruktion/Diffusor01}

}
Um diese zusätzlich zu erhöhen, wird ein Streuplättchen (D:~25~mm) aus transparentem Papier in die gesenkten Kegel geklebt. 

{
\centering\includegraphics[width=0.85\columnwidth]{Abbildungen/Konstruktion/LED02}

}

Zur Fixierung der LED -Matrix werden Winkel aus ABS gedruckt. Diese sorgen dafür, dass die MDF-Platte in Position gehalten wird. Schlussendlich wird noch ein Kabel an die Spanplatte geschraubt, um die Uhr an die Wand zu hängen. Diese Aufhängung ermöglicht ein gerades Ausrichten an nur einem Punkt. 


\end{multicols}

\begin{landscape}
	\begin{figure}
		\centering
		\includegraphics[width=21cm]{Abbildungen/Konstruktion/Grundplatte}
		\caption[Spanplatte]{Fertigungszeichnung Spanplatte}
		\label{fig:Spanplatte}
	\end{figure}
\end{landscape}


